\documentclass[dvipdfmx]{jsarticle}

\title{西暦和暦変換プログラムの作成(Java版)}
\author{Seiichi Nukayama}
\date{2020-05-05}
\usepackage{tcolorbox}
\usepackage{color}
\usepackage{listings, plistings}

% Java
\lstset{% 
  frame=single,
  backgroundcolor={\color[gray]{.9}},
  stringstyle={\ttfamily \color[rgb]{0,0,1}},
  commentstyle={\itshape \color[cmyk]{1,0,1,0}},
  identifierstyle={\ttfamily}, 
  keywordstyle={\ttfamily \color[cmyk]{0,1,0,0}},
  basicstyle={\ttfamily},
  breaklines=true,
  xleftmargin=0zw,
  xrightmargin=0zw,
  framerule=.2pt,
  columns=[l]{fullflexible},
  numbers=left,
  stepnumber=1,
  numberstyle={\scriptsize},
  numbersep=1em,
  language={Java},
  lineskip=-0.5zw,
  morecomment={[s][{\color[cmyk]{1,0,0,0}}]{/**}{*/}},
}
%\usepackage[dvipdfmx]{graphicx}
\usepackage{url}
\usepackage[dvipdfmx]{hyperref}
\usepackage{amsmath, amssymb}
\usepackage{itembkbx}
\usepackage{eclbkbox}	% required for `\breakbox' (yatex added)
\fboxrule=1pt
\parindent=1em
\begin{document}

\pagenumbering{arabic}

%% 修正時刻: Wed May  6 07:18:47 2020


\pagenumbering{arabic}

\section{プログラムの説明}

 \subsection{どういうプログラムにするか}

Javaを使って、西暦・和暦変換プログラムを作ります。

年号は以下のものを使います。

 \begin{table}[h]
  \centering
  \begin{tabular}{|c|c|} \hline
   明治 & 1868 \\ \hline
   大正 & 1912 \\ \hline
   昭和 & 1926 \\ \hline
   平成 & 1989 \\ \hline
   令和 & 2019 \\ \hline
  \end{tabular}
 \end{table}

データを扱うデータ形式ですが、将来的に江戸時代の年号も扱えるようにという
ことで、HashMapを使うことにします。

\begin{thebibliography}{}
 \bibitem{sukkiri} Mapの使い方 『すっきりわかるJava入門 第3版』p633
\end{thebibliography}

\subsection{プログラムの作業場所}

デスクトップか、あるいは適当なフォルダに {\textsf myjava}というフォルダを作っ
て、その中を作業場所とします。

そのフォルダの中に入って、{\textsf Nengo.java}というファイルを作成してください。

とりあえず以下のような内容にしてください。

\begin{lstlisting}
 // Nengo.java

 public class Nengo {

     public static void main (String [] args) {

     }
 }
\end{lstlisting}

\newpage

%%===============================================================================
\section{課題}

%%-----------------------------------------------------------------------
\subsection{課題1}
\begin{breakbox}
 データの内容をすべて出力するようにしてください。順番は年号順でなくてもかまいま
 せん。ハッシュマップは順番は気にしません。
\begin{verbatim}
    > java Nengo
    明治:1867年
    令和:2018年
    昭和:1925年
    平成:1988年
    大正:1911年
    未来:2050年
\end{verbatim}
\end{breakbox}

%%-----------------------------------------------------------------------
\subsection{ヒント1}

{\textsf main}メソッドの処理を考えてみます。
\begin{enumerate}
 \item ハッシュマップ変数を用意する
 \item その変数にデータを入れる
 \item データを出力する
\end{enumerate}

変数を用意するのは、{\textgt p634}を参考に考えます。すると、今回の場合、
以下のようになります。

\begin{tcolorbox}
 {\large Map \textless String, Integer \textgreater \ year = new HashMap
 \textless \textgreater ();}
\end{tcolorbox}

{\textsf year}という変数名にしてみました。それにデータを入れます。

\begin{tcolorbox}
 {\large year.put(``明治'', 1867)}
\end{tcolorbox}

これを全てのデータについて行います。

データを出力するのは、{\textsf p636}のやり方なら、

\begin{tcolorbox}
\begin{verbatim}
     for (String key : year.keyset()) {
       Integer value = year.get(key);
       System.out.println( key + ``:'' + value + ``年'')
     }
\end{verbatim}
\end{tcolorbox}

となります。

そして、データをセットする処理は、たとえば{\textsf setData}メソッドとい
う名前で、データを出力する処理は、たとえば{\textsf printData}メソッドと
いう名前でまとめておきます。

となると、{\textsf year}変数の宣言を記述する場所を考えないといけません。

これをクラス変数とすることで、このクラスの中で参照することができます。

このようになると思います。

%\begin{tcolorbox}
 \begin{verbatim}
    +------ class Nengo -------+
    |  変数year の定義         |
    |  setDataメソッド         |
    |  printDataメソッド       |
    |  mainメソッド            |
    |    + setData();          |
    |    + printData();        |
    +--------------------------+    
 \end{verbatim}
%\end{tcolorbox}

\begin{tcolorbox}
 \begin{verbatim}
    public class Nengo {
        private Map<String, Integer> year;

        public void setData() {
            year = new HashMap <> ();
            // 処理
        }

        public void printData() {
            // 処理
        }

        public static void main (String[] args) {
            Nengo nengo = new Nengo();
            nengo.setData();
            nengo.printData();
        }
    }
 \end{verbatim}
\end{tcolorbox}

{\textsf year}という変数にデータをセットするのは、{\textsf main}が行って
いますが、本来は {\textsf Nengo}というクラスから {\textsf year}というイ
ンスタンスを生成するときに自動的に行う方がよいでしょう。\\*
以下のようにコンストラクタを作っておきます。
\begin{tcolorbox}
 \begin{verbatim}
    public class Nengo {
        private Map<String, Integer> year;

        // コンストラクタ
        public Nengo () {
            this.setData();
        }
        // ... 略 ...
 \end{verbatim}
\end{tcolorbox}




\vspace{2cm}

%%======================================================================

\subsection{課題2}

\begin{breakbox}
 {\textsf \textless java Nengo -n 2003 }とすると、西暦を年号に変換できる
 ようにします。『{\textsf p216}コマンドライン引数』を参照してください。
\begin{verbatim}
    > java Nengo -n 2003
    西暦2003年は 平成15年 です。
\end{verbatim}
\end{breakbox}

%%----------------------------------------------------------

\subsection{ヒント1}

まず、コマンドライン引数を画面に表示することを考えます。

\begin{verbatim}
    > java Nengo -n 2003 
\end{verbatim}
の場合、{\textsf args[0]}に「-n」 が、{\textsf args[1]}に「2003」が入ります。

また、引数の指定が間違っているとき、正しいやり方を表示して、ユーザーに助
言するようにします。

\begin{tcolorbox}
\begin{verbatim}
    > java Nengo 2003
    このように入力してください。
    java Nengo -n 2003
\end{verbatim}
\end{tcolorbox}

\begin{tcolorbox}
\begin{verbatim}
    > java Nengo -n 2003
    args[0]: -n
    args[1]: 2003
\end{verbatim}
\end{tcolorbox}

このようなコードになるでしょう。

\begin{tcolorbox}
\begin{verbatim}
    if (args == 2) {
        // argsの内容を出力
    } else {
        // ユーザーに助言
    }        
\end{verbatim}
\end{tcolorbox}

%%-------------------------------------------------------------

\subsection{ヒント2}

次に、西暦を与えると、和暦を返す関数を考えます。

\begin{verbatim}
    明治:1867年
    大正:1911年
    昭和:1925年
    平成:1988年
    令和:2018年
    未来:2050年
\end{verbatim}

2003年を例に考えると、

\begin{verbatim}
    2003 - 1867 = 明治136年
    2003 - 1911 = 大正92年
    2003 - 1925 = 昭和78年
    2003 - 1988 = 平成15年
    2003 - 2018 = 令和-15年
\end{verbatim}
となります。

% もし、1988年だと、1988 - 1988 = 平成0年 となりますから、
% 引き算をやって 1以上だといいわけです。


順に引き算をやっていって、一番小さな値のときが、その西暦のときの年号だと
いうことになります。

もし、与えた引数が1988年だと、1988 - 1988 = 平成0年 となりますから、
その年号は使えません。引き算をやって 1以上だといいわけです。

実際にコードで記述してみます。

{\textsf p636}のやり方でやってみます。

\begin{verbatim}
    Map <String, Integer> year = new HashMap <> ();
\end{verbatim}

という変数yearを用意しているとして、

ユーザーが入力した西暦を seireki という int型の変数にセットしているとし
て、

\begin{tcolorbox}
\begin{verbatim}
    String nengo = "";  // 求める年号をこれにセットする
    int nen = 0;        // 求める年をこれにセットする    
    int sa = 0;       // 西暦とその年号0年時の西暦年との差
    for (String key : year.keyset()) {
        Integer value : year.get(key);
        sa = seireki - value;
        if (sa > 0) {
            nengo = key;
            nen = sa;
        }   
    }
\end{verbatim}
\end{tcolorbox}

ただ、もとのデータ{\textsf year}が年が昇順に並んでいないので、{\textsf
nen}に最小のデータがセットされるとは限りません。

{\textsf nen}に最小のデータがセットされるように工夫する必要があります。

\begin{tcolorbox}
\begin{verbatim}
    String nengo = "";  // 求める年号をこれにセットする
    int nen = 10000;        // 求める年をこれにセットする    
    int sa = 0;       // 西暦とその年号0年時の西暦年との差
    for (String key : year.keyset()) {
        Integer value : year.get(key);
        sa = seireki - value;
        if (sa > 0 && sa < nen) {
            nengo = key;
            nen = sa;
        }   
    }
\end{verbatim}
\end{tcolorbox}

まず、{\textsf nen}の初期値を大きな値にしておきます。

そして、{\textsf sa}が 0 よりも大きくて、かつ、{\textsf nen}よりも小さけ
れば {\textsf nen}にセットすることにすればいいことになります。




\end{document}

%% 修正時刻: Sat May  2 15:10:04 2020


%% 修正時刻: Tue May 12 06:19:07 2020
