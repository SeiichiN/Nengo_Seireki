
\documentclass[dvipdfmx]{jsarticle}

\include{begin}

\vspace{2cm}

%%======================================================================

\subsection{課題4}

\fboxrule=1pt
\begin{breakbox}
Nengo.java をパッケージ化します。
パッケージ名は ''com.example.nengo'' とします。\\*
現在は ''myjava'' フォルダにいます。ここで、''source''フォルダ
と''classes''フォルダを作成します。\\*
\hspace{1cm} \verb*!> mkdir source classes! \\*
さらに、''source''フォルダの中に
 ''com\yen example\yen nengo'' フォルダを作成します。\\*
\hspace{1cm} \verb*!> mkdir -p com/example/nengo! \\*
 ''com\yen example\yen nengo'' フォルダの中に ''Nengo.java''を移動します。

\begin{tcolorbox}
\begin{verbatim}
  ./myjava
    ├── classes
    └── source
        └── com
            └── example
                └── nengo
                    ├── Nengo.java
                    └── Xnengo.java
\end{verbatim}
\end{tcolorbox}
 
 そして、''Nengo.java''とは別に ''Xnengo.java'' を作成し、その中に
 ''Nengo.java'' の中の ''main''メソッドを移動します。

\begin{tcolorbox}
\begin{verbatim}
    // Xnengo.java
    public class Xnengo {
        public static void main (String[] args) {
          ....      ...
          .... 処理 ...
          ....      ...
        }
    }
\end{verbatim}
\end{tcolorbox}

コンパイルは、''source''フォルダにておこないます。\\*
\hspace{1cm} \verb*!> javac -d ../classes com/example/nengo/*.java! \\*
おそらくエラーが出力されることと思います。


\begin{tcolorbox}
\begin{verbatim}
Xnengo.java:43: エラー: nengoはNengoでprivateアクセスされます
                    + objNengo.nengo + objNengo.nen + "年です。");
                              ^                ^
Xnengo.java:43: エラー: nenはNengoでprivateアクセスされます
                    + objNengo.nengo + objNengo.nen + "年です。");
                              ^                ^
Xnengo.java:49: エラー: seirekiはNengoでprivateアクセスされます
  System.out.println( nengo + nen                            
                    + "年は " + objNengo.seireki + "年です。");
                                        ^
\end{verbatim}
\end{tcolorbox}


 行はコードの書き方によってさまざまですが、\underline{nengoはNengoで
 privateアクセスされます} というエラーが出力されるはずです。
 このエラーに対処してください。
\end{breakbox}



%%----------------------------------------------------------

\subsection{ヒント1}

もしかすると、上にあるようなエラーが出力されないかもしれません。その場合、
以降の内容をみて、以下の対策がされているのかもしれません。

上記のエラーは、''Nengo.java'' の中の以下の部分に関係しています。

\begin{tcolorbox}
\begin{verbatim}
        ...
public class Nengo {
    private HashMap<String, Integer> year;

    private String nengo = "";        --- <1>
    private int nen = 10000;          --- <2>
    private int seireki = 0;          --- <3>
        ...
\end{verbatim}
\end{tcolorbox}

\verb!<1> <2> <3>! は {\textsf private}宣言されてます
(『スッキリわかるJava入門 第3版』p499)。
ですから、
''Nengo''クラス内部から呼び出す(アクセスする)
ぶんには問題ないのですが、今回のように ''main''メソッドを別クラスにして、
別クラスからこのNengoクラスにアクセスすると、今回のようなエラーが出ます。


対処のしかたは、『同書』p503 以降に書かれているように、''getter'' や ''
setter'' を追加します。

''main''メソッドからアクセスする場合は、''getter''メソッドを使えばいいと
いうことになります。

\include{end}

%% 修正時刻: Tue May 12 06:15:05 2020
