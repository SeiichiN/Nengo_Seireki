\documentclass[dvipdfmx]{jsarticle}

\title{西暦和暦変換プログラムの作成(Java版)}
\author{Seiichi Nukayama}
\date{2020-05-05}
\usepackage{tcolorbox}
\usepackage{color}
\usepackage{listings, plistings}

% Java
\lstset{% 
  frame=single,
  backgroundcolor={\color[gray]{.9}},
  stringstyle={\ttfamily \color[rgb]{0,0,1}},
  commentstyle={\itshape \color[cmyk]{1,0,1,0}},
  identifierstyle={\ttfamily}, 
  keywordstyle={\ttfamily \color[cmyk]{0,1,0,0}},
  basicstyle={\ttfamily},
  breaklines=true,
  xleftmargin=0zw,
  xrightmargin=0zw,
  framerule=.2pt,
  columns=[l]{fullflexible},
  numbers=left,
  stepnumber=1,
  numberstyle={\scriptsize},
  numbersep=1em,
  language={Java},
  lineskip=-0.5zw,
  morecomment={[s][{\color[cmyk]{1,0,0,0}}]{/**}{*/}},
}
%\usepackage[dvipdfmx]{graphicx}
\usepackage{url}
\usepackage[dvipdfmx]{hyperref}
\usepackage{amsmath, amssymb}
\usepackage{itembkbx}
\usepackage{eclbkbox}	% required for `\breakbox' (yatex added)
\fboxrule=1pt
\parindent=1em
\begin{document}

\pagenumbering{arabic}

%% 修正時刻: Wed May  6 07:18:47 2020


\subsection{課題3 - 解答例}


\begin{lstlisting}
 // Nengo.java
 
 import java.util.HashMap;
 
 public class Nengo {
     private HashMap<String, Integer> year;
 
     private String nengo = "";
     private int nen = 10000;
     private int seireki = 0;
 
     public Nengo () {
         this.setData();
     }
     
     private void setData() {
         this.year = new HashMap<>();
         this.year.put("明治", 1867);
         this.year.put("大正", 1911);
         this.year.put("昭和", 1925);
         this.year.put("平成", 1988);
         this.year.put("令和", 2018);
         this.year.put("未来", 2050);
     }
 
     public void printData() {
         for ( String key : this.year.keySet() ) {
             int value = this.year.get(key);
             System.out.println( key + ":" + value + "年");
         }
     }
 
     public void toNengo(int seireki) {
         for ( String key : this.year.keySet() ) {
             int value = this.year.get(key);
             int sa = seireki - value;
             if (sa > 0 && sa < this.nen){
                 this.nen = sa;
                 this.nengo = key;
             }
         }
     }
 
     public void toSeireki( String nengo, int nen ) {
         for (String key : this.year.keySet()) {
             int value = this.year.get(key);
             if (key.equals(nengo)) {
                 this.seireki = value + nen;
             }
         }
     }
 
     // forEach文を使うと簡潔に書ける
     public void toSeireki2( String nengo, int nen ) {
         this.year.forEach((k, v) -> {
                 if (k.equals(nengo)) {
                     this.seireki = v + nen;
                 }
             });
     }
 
     public static void main (String[] args) {
         int seireki = 0;      // 西暦の年
         int nen = 0;          // 年号の年
         String nengo = "";    // 年号
         Nengo objNengo = new Nengo();
         // nengo.printData();
         if (args.length == 2 || args.length == 3) {
             ;
         } else {
             System.out.println("このように入力してください。");
             System.out.println("> java Nengo -n 2003  または");
             System.out.println("> java Nengo -g 平成 4");
             System.exit(1);
         }
         if ((args[0]).equals("-n") || args[0].equals("-g")) {
             ;
         } else {
             System.out.println("第1引数は -n あるいは -g です。");
             System.out.println("-n : 年号を求める場合");
             System.out.println("-g : 西暦を求める場合");
             System.exit(1);
         }
         if (args[0].equals("-n")) {
             seireki = Integer.parseInt(args[1]);
             if (seireki < 1867) {
                 System.out.println("入力できる西暦は1868年以上です。");
                 System.exit(1);
             }
             if (seireki > 2050) {
                 System.out.println("入力できる西暦は2050年未満です。");
                 System.exit(1);
             }
  
             objNengo.toNengo(seireki);
             System.out.println("西暦" + seireki + "年は "
                                + objNengo.nengo + objNengo.nen + "年です。");
         }
         if (args[0].equals("-g")) {
             nengo = args[1];
             nen = Integer.parseInt(args[2]);
             objNengo.toSeireki( nengo, nen );
             System.out.println( nengo + nen + "年は " + objNengo.seireki + "年です。");
         }
     }
 }
\end{lstlisting}



\end{document}

%% 修正時刻: Sat May  2 15:10:04 2020


%% 修正時刻: Tue May 12 05:48:19 2020
