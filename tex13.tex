\documentclass[dvipdfmx]{jsarticle}

\include{begin}

\subsection{課題4 - 解答例}

\begin{lstlisting}[caption=com/example/nengo/Nengo.java]
  ...
 public class Nengo {
    private HashMap<String, Integer> year;

    private String nengo = "";
    private int nen = 10000;
    private int seireki = 0;

    public Nengo () {
        this.setData();
    }
    
    // getter
    public String getNengo () { return this.nengo; }
    public int getNen () { return this.nen; }
    public int getSeireki () { return this.seireki; }

    // setter -- 今回は使わない 
    public void setNengo (String newNengo) { this.nengo = newNengo; }
    public void setNen (int newNen) { this.nen = newNen; }
    public void setSeireki (int newSeireki) { this.seireki = newSeireki; }
   ...
\end{lstlisting}

以上のように ''getter'' を作ります。''setter'' も作っておいてもいいです。
今のところ使っていませんが。

Xnengo.java も Nengoインスタンスの nengo、nen、seireki にアクセスしてい
るところを、上で作ったゲッターを使います。

 \begin{lstlisting} [caption=com/example/nengo/Xnengo.java]
  ...
 System.out.println("西暦" + seireki + "年は "
     + objNengo.getNengo() + objNengo.getNen() + "年です。");
  ...
  System.out.println( nengo + nen + "年は " + objNengo.getSeireki() + "
  年です。");
  ...
 \end{lstlisting}

 
\include{end}

%% 修正時刻: Tue May 12 05:48:30 2020
