\documentclass[dvipdfmx]{jsarticle}

\title{西暦和暦変換プログラムの作成(Java版)}
\author{Seiichi Nukayama}
\date{2020-05-05}
\usepackage{tcolorbox}
\usepackage{color}
\usepackage{listings, plistings}

% Java
\lstset{% 
  frame=single,
  backgroundcolor={\color[gray]{.9}},
  stringstyle={\ttfamily \color[rgb]{0,0,1}},
  commentstyle={\itshape \color[cmyk]{1,0,1,0}},
  identifierstyle={\ttfamily}, 
  keywordstyle={\ttfamily \color[cmyk]{0,1,0,0}},
  basicstyle={\ttfamily},
  breaklines=true,
  xleftmargin=0zw,
  xrightmargin=0zw,
  framerule=.2pt,
  columns=[l]{fullflexible},
  numbers=left,
  stepnumber=1,
  numberstyle={\scriptsize},
  numbersep=1em,
  language={Java},
  lineskip=-0.5zw,
  morecomment={[s][{\color[cmyk]{1,0,0,0}}]{/**}{*/}},
}
%\usepackage[dvipdfmx]{graphicx}
\usepackage{url}
\usepackage[dvipdfmx]{hyperref}
\usepackage{amsmath, amssymb}
\usepackage{itembkbx}
\usepackage{eclbkbox}	% required for `\breakbox' (yatex added)
\fboxrule=1pt
\parindent=1em
\begin{document}

\pagenumbering{arabic}

%% 修正時刻: Wed May  6 07:18:47 2020


\subsection{課題4 - 解答例}

\begin{lstlisting}[caption=com/example/nengo/Nengo.java]
  ...
 public class Nengo {
    private HashMap<String, Integer> year;

    private String nengo = "";
    private int nen = 10000;
    private int seireki = 0;

    public Nengo () {
        this.setData();
    }
    
    // getter
    public String getNengo () { return this.nengo; }
    public int getNen () { return this.nen; }
    public int getSeireki () { return this.seireki; }

    // setter -- 今回は使わない 
    public void setNengo (String newNengo) { this.nengo = newNengo; }
    public void setNen (int newNen) { this.nen = newNen; }
    public void setSeireki (int newSeireki) { this.seireki = newSeireki; }
   ...
\end{lstlisting}

以上のように ''getter'' を作ります。''setter'' も作っておいてもいいです。
今のところ使っていませんが。

Xnengo.java も Nengoインスタンスの nengo、nen、seireki にアクセスしてい
るところを、上で作ったゲッターを使います。

 \begin{lstlisting} [caption=com/example/nengo/Xnengo.java]
  ...
 System.out.println("西暦" + seireki + "年は "
     + objNengo.getNengo() + objNengo.getNen() + "年です。");
  ...
  System.out.println( nengo + nen + "年は " + objNengo.getSeireki() + "
  年です。");
  ...
 \end{lstlisting}

 
\end{document}

%% 修正時刻: Sat May  2 15:10:04 2020


%% 修正時刻: Tue May 12 05:48:30 2020
