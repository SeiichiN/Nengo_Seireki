\documentclass[dvipdfmx]{jsarticle}

\title{西暦和暦変換プログラムの作成(Java版)}
\author{Seiichi Nukayama}
\date{2020-05-05}
\usepackage{tcolorbox}
\usepackage{color}
\usepackage{listings, plistings}

% Java
\lstset{% 
  frame=single,
  backgroundcolor={\color[gray]{.9}},
  stringstyle={\ttfamily \color[rgb]{0,0,1}},
  commentstyle={\itshape \color[cmyk]{1,0,1,0}},
  identifierstyle={\ttfamily}, 
  keywordstyle={\ttfamily \color[cmyk]{0,1,0,0}},
  basicstyle={\ttfamily},
  breaklines=true,
  xleftmargin=0zw,
  xrightmargin=0zw,
  framerule=.2pt,
  columns=[l]{fullflexible},
  numbers=left,
  stepnumber=1,
  numberstyle={\scriptsize},
  numbersep=1em,
  language={Java},
  lineskip=-0.5zw,
  morecomment={[s][{\color[cmyk]{1,0,0,0}}]{/**}{*/}},
}
%\usepackage[dvipdfmx]{graphicx}
\usepackage{url}
\usepackage[dvipdfmx]{hyperref}
\usepackage{amsmath, amssymb}
\usepackage{itembkbx}
\usepackage{eclbkbox}	% required for `\breakbox' (yatex added)
\fboxrule=1pt
\parindent=1em
\begin{document}

\pagenumbering{arabic}

%% 修正時刻: Wed May  6 07:18:47 2020


\vspace{2cm}

%%======================================================================

\subsection{課題3}

\begin{breakbox}
 \underline{{\textsf \textgreater \, java Nengo -g 昭和 41年 }} あるいは \underline{{\textsf
 \textgreater \, java Nengo -g 昭和 41 }}とすると、年号を西暦に変換できる
 ようにします。『{\textsf p216}コマンドライン引数』を参照してください。
\begin{verbatim}
    > java Nengo -g 昭和 41
    昭和41年は 西暦1966年 です。
\end{verbatim}
\end{breakbox}

%%-----------------------------------------------------------------------
\subsection{ヒント1}

{\textsf Nengo.java}の {\textsf main}メソッドの引数処理の部分を修正しま
す。

\begin{itemize}
 \item {\textsf args.length}が {\textsf 2} あるいは {\textsf 3} なら
       {\textsf OK} とします。
 \item 第1引数が {\textsf -n} あるいは {\textsf -g} 以外は、ユーザーに対
       してアドバイスします。
 \item 第1引数が {\textsf -n} の場合、第2引数のチェックと取り込みをしま
       す。そして {\textsf toNengo}メソッドに処理を渡します。
 \item 第1引数が {\textsf -g} の場合、第2引数と第3引数の取り込みをします。
       チェックはしません。そして、新しく {\textsf toSeireki}メソッドを
       作成し、それに処理を渡します。
\end{itemize}

{\textsf toSeireki}メソッドは、まだ空状態で {\textsf OK} です。

%%-----------------------------------------------------------------------
\subsection{ヒント2}

次に toSeirekiメソッドを作っていきます。

toSeirekiメソッドの引数は、String型の nengo と int型
の nen とします。

処理の結果は 西暦の整数です。これは、クラス変数 seireki としておきます。

処理の内容は、HashMap型クラス変数の this.year の
nengo と 引数の nengo を比べていき、同じなら、
this.year.nen と nen を合計した数が求める西暦です。

たとえば、ユーザーが「平成 2」と入力したとします。

this.year の nengo と比べていき、this.year
の {\em 平成} と一致したとき、this.year の nen は
1988 です。

1988 + 2 = 1990。つまり、西暦1990年です。

ここまでをコードにすると、以下のようになります。

\begin{lstlisting}
 public void toSeireki( String nengo, int nen ) {
     for (String key : this.year.keySet()) {
         int value = this.year.get(key);
         if (key.equals(nengo)) {
             this.seireki = value + nen;
         }
     }
 }
\end{lstlisting}

ただ、この for文はごちゃごちゃしてて見にくいので、次の forEach文のほうが
すっきりしてます。

\begin{lstlisting}
 public void toSeireki( String nengo, int nen ) {
     this.year.forEach((key, value) -> {
         if (key.equals(nengo)) {
             this.seireki = value + nen;
         }
     });
 }
\end{lstlisting}


\end{document}

%% 修正時刻: Sat May  2 15:10:04 2020


%% 修正時刻: Tue May 12 06:14:38 2020
